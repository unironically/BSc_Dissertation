\documentclass[a4paper, 11pt]{article}
\usepackage{pgfgantt}
\usepackage{fancyhdr}
\usepackage{textcomp}


\fancypagestyle{title}{
\renewcommand{\headrulewidth}{0pt}
\fancyhf{}
\lhead{Luke Bessant - 2019}
\rhead{}
}
\pagestyle{title}

\title{\textbf{CS3821 Full Unit Project}\\On Context-Free Grammar Specification}
\author{Luke Bessant\\Supervisor: Reuben Rowe}
\date{\today}

\begin{document}

\maketitle
\thispagestyle{title}
\newpage

\tableofcontents
\newpage

\section{Introduction}
A context-free grammar is a language specification which allows a computer to derive the desired grammatical structure intended by the programmer to a list of tokens which make up a statement, giving a computer the ability to understand its meaning and therefore generate the correct syntax tree for the statement which is then used in later stages of compilation or interpretation.

\section{Context-Free Grammars}
We use context-free grammars to specify the way in which strings of the language we want to process should be structured, using patterns which can later be identified within the list of tokens created during the lexing of the source code being processed. Context-free grammars describe the hierarchical structure of language constructs such as statements and expressions which are commonly used in program code. Take for instance the following:

\begin{center}
	\textit{statement} \textbf{\textrightarrow} \texttt{while}(\textit{expression})\texttt{:} \textit{statement}
\end{center}

Represented here is a production rule for a \textit{statement}, in this example a simple one line Python while loop which will execute the code represented by \textit{statement} until the variable \textit{expression} is false. The string on the right side is a concatenation of the keyword \textbf{while}, a terminal, an opening parenthesis, a variable \textit{expression}, a closing bracket, a colon, and a variable \textit{statement}. We can add in the below two production rules to our grammar and assign the nonterminal 'statement' as the start symbol.

\begin{center}
	\begin{tabular}{l}
		\textit{expression} \textbf{\textrightarrow} \texttt{True} \\
		\textit{statement} \textbf{\textrightarrow} \texttt{print(}\textit{expression}\texttt{)}
	\end{tabular}
\end{center}

To allow us to generate the below string ``while(True): print(True)", belonging to the language generated by this grammar using variables and terminals. The definitions for variables, which we call \textit{nonterminals}, terminals and production rules will be outlined in the following section.

\newpage
\section{Specification}
A correctly specified context-free grammar must consist of the following components:

\begin{itemize}
	\item A set of \textit{terminal} symbols. These are characters or keywords which make up the strings of the language, e.g. \textit{[a..z]}, \textit{[0..9]} or mathematical operators. The terminals within a context-free grammar make up the \textit{alphabet} of the language generated.

	\item A set of \textit{nonterminals} or \textit{syntactic variables} which represent the strings which can be derived using their production rules. Each makes up the head of at least one production rule, describing the strings that can be derived from it, which must always result in a string of zero or more terminal symbols.

	\item A list of construction rules known as \textit{productions}, dictating the strings that can be created from a nonterminal symbol. A nonterminal is at the \textit{head} (left side) of the production, separated by the symbol \textbf{\textrightarrow} or \textbf{::=} as ``has the form", from the \textit{body} of the production (right side), consisting of a pattern of zero or more terminals and nonterminals. The body represents one of the written forms of a nonterminal.

	\item Identification of the \textit{start symbol}, a nonterminal, usually the first listed or otherwise designated visually, from which we can derive all of the strings which make up the language generated by the grammar.
\end{itemize}

\subsection{Example}
An example for a simple two function calculator context-free grammar specification is shown below:

\begin{center}
	\begin{tabular}{l}
		\textit{operand} \textbf{\textrightarrow} \textit{operand}\texttt{+}\textit{operand} \\
		\textit{operand} \textbf{\textrightarrow} \textit{operand}\texttt{-}\textit{operand} \\
		\textit{operand} \textbf{\textrightarrow} \texttt{a}
	\end{tabular}
\end{center}

The terminals \texttt{a, -} and \texttt{+} are shown in bold, and the nonterminals are italicised. To note, we can use the ``\textbar"\ symbol to combine multiple productions with the same head into one production as below:

\begin{center}
	\begin{tabular}{l}
		\textit{operand} \textbf{\textrightarrow} \textit{operand}\texttt{+}\textit{operand} \textbar\ \textit{operand}\texttt{-}\textit{operand} \textbar\ \texttt{a}
	\end{tabular}
\end{center}

Some examples of the strings contained within the language generated by this grammar are \texttt{a, a+a, a+a-a} and \texttt{a+a+a+a}. 

\newpage
\section{Derivations}
We generate strings belonging to the language of a grammar by beginning with the start symbol and replacing the nonterminals its body with the body of a production for that nonterminal, eventually ending up with a string which is an element of the language. For the grammar specified in Section 3.1 with the start symbol being our only nonterminal, \textit{operand}, we can replace this symbol with \textit{operand}\texttt{+}\textit{operand}, \textit{operand}\texttt{-}\textit{operand} or \texttt{a}.
\\\\
We show that a nonterminal A \textit{derives} a terminal or other nonterminal B by $A \Rightarrow B$. This is also called a replacement. We can show the derivation of a particular string by replacements. Using the example grammar in Section 3.1:

\begin{center}
	$operand \Rightarrow operand+operand$ $\Rightarrow \texttt{a}+operand$ $\Rightarrow \texttt{a+a}$
\end{center}

This shows the \textit{derivation} of \texttt{a+a} from \textit{operand}, proving that \texttt{a+a} is a string of the grammar from Section 3.1. During parsing we take a string of terminal symbols such as the one shown above and then try to figure out how to derive it from the start symbol.

\section{Backus–Naur Form Notation}
Backus–Naur form or BNF for short is simply a language for creating context-free grammar specifications, thus such specifications have near-identical structure to the previous examples of context-free grammars, consisting of a head, a ``has the form" symbol (usually \texttt{::=}) and a body. Thus every rule in BNF takes the form:

\begin{center}
	\begin{tabular}{l}
		\textit{head} \texttt{::=} \textit{body}
	\end{tabular}
\end{center}

Generally nonterminals are enclosed within angle brackets, such as ``\textlangle{}operand\textrangle{}", whereas terminal symbols are enclosed in double or single quotation marks. Therefore we can define the grammar from Section 3.1 in BNF with the following syntax:

\begin{center}
	\texttt{<operand> ::= <operand> '+' <operand> \textbar\ <operand> '-' <operand> \textbar\ 'a'}
\end{center}

We can also surround certain contents of the body of a BNF production in curly braces to indicate zero or more repetitions of the contents. For example, when specifying functions which take a list of parameters we can use \texttt{\{<parameter>\}} within the body to indicate zero or more parameters. The example shown above also defines a \textit{recursive statement} since the production rule makes use of the head nonterminal in its body. In conclusion, the use of BNF is to give us a standard syntactic structure for specifying context-free grammars.

\newpage
\addcontentsline{toc}{section}{References}
\begin{thebibliography}{9}

\bibitem{Aho} 
Aho, A. Lam, M. Sethi, R. Ullman, J.
[\textit{Compilers - Principles, Techniques, and Tools: Second Edition}]. 
2007.

\bibitem{Johnstone} 
Johnstone, A.
[\textit{CS3480 Software Engineering with Metamodels}]. 
2013.

\bibitem{Might} 
Might, M.
[\textit{The language of languages}]. 

\texttt{http://matt.might.net/articles/grammars-bnf-ebnf/}

\bibitem{teach-ict} 
www.teach-ict.com
[\textit{Backus-Naur Form (BNF)}]. 

\texttt{https://www.teach-ict.com/as\_as\_computing/ocr/H447/3\_3\_2
/lexical\_syntax\_analysis/miniweb/pg8.htm}


\end{thebibliography}

\end{document}
